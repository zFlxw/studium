\documentclass[11pt]{article}
\usepackage[utf8]{inputenc}
\usepackage{amsmath,amsthm,amsfonts,amssymb,amscd}
\usepackage{multirow,booktabs}
\usepackage{xcolor}
\usepackage{fullpage}
\usepackage{lastpage}
\usepackage{enumitem}
\usepackage{fancyhdr}
\usepackage{mathrsfs}
\usepackage{array}
\usepackage{wrapfig}
\usepackage{setspace}
\usepackage{calc}
\usepackage{multicol}
\usepackage{cancel}
\usepackage[margin=3cm]{geometry}
\usepackage[retainorgcmds]{IEEEtrantools}
\usepackage[most]{tcolorbox} \usepackage{xcolor}

\usepackage{minted}
\usepackage{booktabs}
\usepackage{hyperref}


\newcommand*{\figref}[2][]{%
  \hyperref[{fig:#2}]{%
    \textit{Fig.~\ref*{fig:#2}}%
    \ifx\\#1\\%
    \else
      \,#1%
    \fi
  }%
}

\newlength{\tabcont}
\setlength{\parindent}{0.0in} \setlength{\parskip}{0.05in} \usepackage{empheq} \usepackage{framed}


\colorlet{shadecolor}{orange!15}
\parindent 0in
\parskip 12pt \geometry{margin=1in, headsep=0.25in} \theoremstyle{definition}

\begin{document}

\thispagestyle{empty}

\newtheorem{anmk}{Anmerkung}[section]
\newtheorem{bsp}{Beispiel}[section]
\newtheorem{definition}{Definition}[section]

\renewcommand{\figurename}{Fig.}

\newcommand{\ol}[1]{\begin{enumerate}#1\end{enumerate}}
\newcommand{\ul}[1]{\begin{itemize}#1\end{itemize}}
\newcommand{\li}[1]{\item{#1}}
\newcommand{\equivto}{\Longleftrightarrow}
\newcommand{\sube}{\subseteq}

\newcommand{\N}{\mathbb{N}}
\newcommand{\Z}{\mathbb{Z}}
\newcommand{\R}{\mathbb{R}}

\begin{center}
  {\LARGE \bf Algorithmen und Komplexität}\\
  {\Large Algorithmen und Komplexität}\\
  WS 2024
\end{center}

\section{Begriffsdefinitionen}
\begin{definition}
  Ein \textbf{Algorithmus} ist eine endliche Folge von Anweisungen, die eine bestimmte Eingabe in eine bestimmte Ausgabe
  umwandelt.
\end{definition}
\begin{definition}
  Eine \textbf{Datenstruktur} dient zur Organisation von Daten. Dabei unterscheidet sich der Fokus von Datenstruktur zu 
  Datenstruktur; manche sind effizienter für bestimmte Operationen als andere.
\end{definition}

\section{Laufzeit}
Als \textbf{Laufzeit} bezeichnet man die Zeit, die ein Programm benötigt, um eine bestimmte Aufgabe zu erledigen. Konrekt bedeutet
das hier, dass die Laufzeit die Zeit beschreibt, die ein Algorithmus benötigt, um eine bestimmte Eingabe zu vearbeiten und
daraus eine bestimmte Ausgabe zu erzeugen.

Da es jedoch nicht immer möglich ist, die Laufzeit eines Algorithmus exakt zu bestimmen, wird die Laufzeit von Algorithmen
in der Regel \textbf{asymptotisch} betrachtet. Das bedeutet, man betrachtet die Laufzeit eines Algorithmus für sehr große
Eingaben und versucht, eine obere Schranke für die Laufzeit zu finden.

\subsection{Tilde-Notation}
Eine mögliche Notation, um die asymptotische Laufzeit eines Algorithmus zu beschreiben, ist die \textbf{Tilde-Notation}.
Diese ist eine simple Annäherung, bei der lediglich der höchste Exponent der Laufzeitfunktion betrachtet wird, da man davon
ausgeht, dass bei großen $N$ die anderen Terme der Funktion vernachlässigbar klein werden.

\begin{bsp}
  Sei $T(n) = 3n^2 + 2n + 1$ die Laufzeitfunktion eines Algorithmus. Dann gilt in Tilde-Notation:
  \begin{align*}
    T(n) = \sim 3n^3
  \end{align*}
\end{bsp}

\subsection{}
\end{document}