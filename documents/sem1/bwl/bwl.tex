\documentclass[11pt]{article}
\usepackage[utf8]{inputenc}
\usepackage{amsmath,amsthm,amsfonts,amssymb,amscd}
\usepackage{multirow,booktabs}
\usepackage{xcolor}
\usepackage{fullpage}
\usepackage{lastpage}
\usepackage{enumitem}
\usepackage{fancyhdr}
\usepackage{mathrsfs}
\usepackage{array}
\usepackage{wrapfig}
\usepackage{setspace}
\usepackage{calc}
\usepackage{multicol}
\usepackage{cancel}
\usepackage[margin=3cm]{geometry}
\usepackage[retainorgcmds]{IEEEtrantools}
\usepackage[most]{tcolorbox} \usepackage{xcolor}

\usepackage{minted}
\usepackage{booktabs}
\usepackage{hyperref}


\newcommand*{\figref}[2][]{%
  \hyperref[{fig:#2}]{%
    \textit{Fig.~\ref*{fig:#2}}%
    \ifx\\#1\\%
    \else
      \,#1%
    \fi
  }%
}

\newlength{\tabcont}
\setlength{\parindent}{0.0in} \setlength{\parskip}{0.05in} \usepackage{empheq} \usepackage{framed}


\colorlet{shadecolor}{orange!15}
\parindent 0in
\parskip 12pt \geometry{margin=1in, headsep=0.25in} \theoremstyle{definition}

\begin{document}

\thispagestyle{empty}

\newtheorem{anmk}{Anmerkung}[section]
\newtheorem{bsp}{Beispiel}[section]
\newtheorem{definition}{Definition}[section]
\newtheorem{aufgabe}{Aufgabe}[section]

\renewcommand{\figurename}{Fig.}

\newcommand{\ol}[1]{\begin{enumerate}#1\end{enumerate}}
\newcommand{\ul}[1]{\begin{itemize}#1\end{itemize}}

\begin{center}
  {\LARGE \bf Betriebswirtschaftlehre}\\
  {\Large Betriebswirtschaftslehre}\\
  WS 2024
\end{center}

\section{UE3 - Übungsaufgaben}
\begin{aufgabe}
  Erläutern Sie die Ziele eines Business Process Reengineering.
  \ul{
    \item Gewinnsituation des Unternehmens verbessern (auf Kundenwünsche z.B.)
  }
\end{aufgabe}

\begin{aufgabe}
  Das Business Process Reengineering ist in der Praxis durch die 4 "R" gekennzeichnet. Beschreiben Sie die 4 R und erklären
  Sie anhand eines Beispiels was damit jeweils bewirkt werden soll.
  \ol{
    \item Renewing: Mitarbeiter sollen gezielt in den Erneuerungsprozess einbezogen werden und entsprechende Kompetenzen erlangen
    \item Revitalizing: Prozessanalyse mit anschließender Begutachtung und Belebung der Abläufe
    \item Reframing: Eingefahrene und vorherrschende Denkmuster werden zerschlagen und ein grundlegendes Umdenken wird in
          die Wege geleitet
    \item Restructuring: Die neuen Prozessansätze werden im gesamten Unternehmen umgesetzt
  }
\end{aufgabe}
\end{document}