\documentclass[11pt]{article}
\usepackage[ngerman]{babel}
\usepackage[utf8]{inputenc}
\usepackage[T1]{fontenc}
\usepackage{amsmath,amsthm,amsfonts,amssymb,amscd}
\usepackage{multirow,booktabs}
\usepackage[table]{xcolor}
\usepackage{fullpage}
\usepackage{lastpage}
\usepackage{enumitem}
\usepackage{fancyhdr}
\usepackage{mathrsfs}
\usepackage{array}
\usepackage{wrapfig}
\usepackage{setspace}
\usepackage[retainorgcmds]{IEEEtrantools}
\usepackage[margin=3cm]{geometry}
\usepackage{amsmath}
\usepackage[most]{tcolorbox} \usepackage{xcolor}

\newlength{\tabcont}
\setlength{\parindent}{0.0in} \setlength{\parskip}{0.05in} \usepackage{empheq} \usepackage{framed}

\colorlet{shadecolor}{orange!15}

\begin{document}

\thispagestyle{empty}

\newcommand{\ol}[1]{\begin{enumerate}#1\end{enumerate}}
\newcommand{\ul}[1]{\begin{itemize}#1\end{itemize}}
\begin{center}
  {\LARGE \bf Workflowmanagement}\\
  {\Large Six Sigma Methode}\\
  WS 2024
\end{center}

\section{Definition}
\ul{
  \item Managementsystem zur Prozessverbesserung
  \item Methode zum Qualitätsmanagement
}

\section{Geschichtlicher Hintergrund}
\ul{
  \item Entwickelt von Motorola in den 1980er Jahren
  \item Erzielte große Popularität nach Einsatz bei General Electric Inc. Ende der 1990er Jahre
  \item Seit 2001 oft in Kombination mit Lean Management $\Longrightarrow$ "Lean Six Sigma"
  \item 2011 erschien mit ISO 13053 die erste internationale Norm für Six Sigma
}

\section{Vorgehensweise}
\ul{
  \item Kernprozess: DMAIC-Zyklus
  \item \textbf{D} -- Define
  \ul{
    \item Projekt-Charta festlegen
    \ul{
      \item Projektziele festlegen (Wunschzustand dokumentieren)
      \item Vermutungen über Ursachen für derzeitige Probleme formulieren
      \item Projektdefinition (Mitglieder, benötigte Ressourcen, Zeitumfang)
    }
    \item Meist weitere Werkzeuge wie
    \ul{
      \item Problemdefinition unter Kepner-Tregoe-Analyse
      \item SIPOC-Diagramm (Supplier, Input, Process, Output, Customer)
      \item CQT-Bäume (Critical to Quality): Bestimmung der messbaren, kritischen Qualitätsmerkmale
      \item VoC (Voice of Customer): Kundenanforderungen mit einbeziehen
    }
  }
  \item \textbf{M} -- Measure
  \ul{
    \item Wie gut deckt der Prozess die Kundenanforderungen ab?
    \item Prozessfähigkeitsuntersuchung
    \ul{
      \item Prozessvisualisierung durch Process Mapping
      \item Statistische Datenherbungs- und Versuchsplanung
    }
  }
  \item \textbf{A} -- Analyze
  \ul{
    \item Warum erfüllt der Prozess die Kundenanforderungen noch nicht (im gewünschten Umfang)?
    \item Möglich Werkzeuge
    \ul{
      \item C\&E Matrix (Cause and Effect): Ursachen-Wirkungs-Diagramm
      \item Durchlaufzeitanalyse
      \item Hypothesentest
      \item Ishikawa-Diagramm (Diagramm zur Darstellung von Ursachen und deren Wirkungen)
      \item Paretodiagramm
      \item Regressionsanalyse
      \item Streudiagramm (Scatter Plot)
      \item Wertschöpfungsanalyse 
    }
  }

  \item \textbf{I} -- Improve
  \ul{
    \item Implementierung und Test des Prozesses
    \item Verwendete Werkzeuge:
    \ul{
      \item Platzzifferverfahren
      \item K.-o.-Analyse
      \item Kriterienbasierte Matrix
      \item Kosten-Nutzen-Analyse
      \item Soll-Prozessdarstellung
      \item Poka Yoke
      \item Brainstorming zum Finden von Verbesserungsmöglichkeiten
      \item FMEA (Failure Mode and Effects Analysis): Methode zur Risikobewertung
    }
  }
  \item \textbf{C} -- Control
  \ul{
    \item Überwachung des Prozesses mithilfe statistischer Methoden
    \item Werkzeuge
    \ul{
      \item SPC-Regelkarten (Statistical Process Control)
      \item Prozessdokumentation
      \item Prozessmanagement- und Reaktionsplan
      \item Precontrol
      \item Projekterfolgsberechnung
    }
  }
}

\section{Rollen und Aufgaben}
\ul{
  \item Richtet sich nach Rangkennzeichen (ähnlich zu Gürtelfarben im Kampfsport)
  \item Niedrigster Rang: grüner Gürtel
  \ul{
    \item Meist Abteilungsleiter, Gruppenleiter oder Planer
  }
  \item Höherer Rang: schwarzer Gürtel
  \ul{
    \item Übernimmt Projektmanagementaufgaben und hat umfassende Kenntnisse in Six Sigma und wie man die Methoden anwendet
    \item Faustregel: vier Verbesserungsmethoden pro Jahr mit insgesamt ca. 200.000 Euro Einsparung
  }
  \item Nächsthöherer Rang: Schwarzer Meistergürtel
  \ul{
    \item Ist Vollzeitverbesserungsexperte
    \item Ist für Coaching und Ausbildung neuer Six Sixma Gürtel verantwortlich
  }
  \item Höchster Rang: Champion
  \ul{
    \item In drei Klassen unterteilt
    \item Leiter der strategischen Managements: langjähriger Unternehmer und leitet Lehrveranstaltungen an Universitäten
    \item Auslieferungschampion: Mitglied der Unternehmensleitung; Leiter von Six Sigma Projekten im Unternehmensleitung
    \item Projektmanagement: Mitglied des mittleren Managements und Auftraggeber für einzelne Six Sigma Projekte
  }
}

\section{Statistische Qualitätsziel}
\ul{
  \item 6 Sigma = 6 Standardabweichungen vom Mittelwert
  \item 3,4 Fehler pro Million
  \item 99,99966\% Fehlerfreiheit
}
\end{document}