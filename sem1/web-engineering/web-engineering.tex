\documentclass[11pt]{article}
\usepackage[utf8]{inputenc}
\usepackage{amsmath,amsthm,amsfonts,amssymb,amscd}
\usepackage{multirow,booktabs}
\usepackage[table]{xcolor}
\usepackage{fullpage}
\usepackage{lastpage}
\usepackage{enumitem}
\usepackage{fancyhdr}
\usepackage{mathrsfs}
\usepackage{array}
\usepackage{wrapfig}
\usepackage{setspace}
\usepackage{calc}
\usepackage{multicol}
\usepackage{cancel}
\usepackage[retainorgcmds]{IEEEtrantools}
\usepackage[margin=3cm]{geometry}
\usepackage{amsmath}
\usepackage[most]{tcolorbox} \usepackage{xcolor}

\newlength{\tabcont}
\setlength{\parindent}{0.0in} \setlength{\parskip}{0.05in} \usepackage{empheq} \usepackage{framed}


\colorlet{shadecolor}{orange!15}
\parindent 0in
\parskip 12pt \geometry{margin=1in, headsep=0.25in} \theoremstyle{definition}

\begin{document}

\thispagestyle{empty}

\newtheorem{definition}{Definition}[section]
\newtheorem{anmk}{Anmerkung}[section]
\newtheorem{bsp}{Beispiel}[section]

\newcommand{\N}{\mathbb{N}}
\newcommand{\Z}{\mathbb{Z}}
\newcommand{\R}{\mathbb{R}}

\begin{center}
  {\LARGE \bf Web Engineering I}\\
  {\Large Web Engineering I}\\
  WS 2024
\end{center}

\section{Aufgabe 1}
\subsection{Aufgabe 1.1}
Die grobe Timeline des Internets sieht folgendermaßen aus:
\begin{itemize}
  \item 1968: Entwicklung des Arpanets durch Forscher des MIT und des US-Verteidigungsministeriums
  \item 1983: Einführung des TCP/IP-Protokolls
  \item 1989: Entwicklung des World Wide Web durch Tim Berners-Lee
  \item 1993: Einführung des ersten Web-Browsers Mosaic (neben dem "World Wide Web" Browser von Berners-Lee)
\end{itemize}

\subsection{Aufgabe 1.2}
Wichtige Entwicklungen und Ereignisse des Internets:
\begin{itemize}
  \item 1972: Erstes E-Mail-Programm wird durch Ray Tomlinson entwickelt
  \item 1977: TCP/IP wird auf Basis des CYCLADES-Netzwerks entwickelt
  \item 1984: Erstmalige Verwendung des Domain Name Systems (DNS)
  \item 1986: Die ersten .de Domains werden registriert
  \item 1998: Google wird gegründet
  \item 1999: Eine Million .de Domains werden registriert
  \item 2001: Wikipedia wird gegründet
\end{itemize}

\subsection{Aufgabe 1.3}
Ein Pionier des Internets ist Teus Hagen, unter anderem an der Entwicklung des TCP/IP-Protokolls beteiligt
war.

\subsection{Aufgabe 1.4}
\begin{itemize}
  \item ISOC: Internet Society. Sie ist eine internationale Organisation, die sich für die Entwicklung und
        Standardisierung des Internets einsetzt.
  \item W3C: World Wide Web Consortium. Es ist eine internationale Organisation, die sich für die Entwicklung und 
  Standardisierung des World Wide Web einsetzt.
  \item ICANN: Internet Corporation for Assigned Names and Numbers. Sie ist eine internationale Organisation, die sich 
  um die Vergabe von Domainnamen und IP-Adressen kümmert.
\end{itemize}
\end{document}