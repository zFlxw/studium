\documentclass[11pt]{article}
\usepackage[utf8]{inputenc}
\usepackage{amsmath,amsthm,amsfonts,amssymb,amscd}
\usepackage{multirow,booktabs}
\usepackage{xcolor}
\usepackage{fullpage}
\usepackage{lastpage}
\usepackage{enumitem}
\usepackage{fancyhdr}
\usepackage{mathrsfs}
\usepackage{array}
\usepackage{wrapfig}
\usepackage{setspace}
\usepackage{calc}
\usepackage{multicol}
\usepackage{cancel}
\usepackage[margin=3cm]{geometry}
\usepackage[retainorgcmds]{IEEEtrantools}
\usepackage[most]{tcolorbox} \usepackage{xcolor}

\usepackage{minted}
\usepackage{booktabs}
\usepackage{hyperref}


\newcommand*{\figref}[2][]{%
  \hyperref[{fig:#2}]{%
    \textit{Fig.~\ref*{fig:#2}}%
    \ifx\\#1\\%
    \else
      \,#1%
    \fi
  }%
}

\newlength{\tabcont}
\setlength{\parindent}{0.0in} \setlength{\parskip}{0.05in} \usepackage{empheq} \usepackage{framed}


\colorlet{shadecolor}{orange!15}
\parindent 0in
\parskip 12pt \geometry{margin=1in, headsep=0.25in} \theoremstyle{definition}

\begin{document}

\thispagestyle{empty}

\newtheorem{anmk}{Anmerkung}[section]
\newtheorem{bsp}{Beispiel}[section]

\renewcommand{\figurename}{Fig.}

\newcommand{\ol}[1]{\begin{enumerate}#1\end{enumerate}}
\newcommand{\ul}[1]{\begin{itemize}#1\end{itemize}}
\newcommand{\li}[1]{\item{#1}}
\newcommand{\equivto}{\Longleftrightarrow}
\newcommand{\sube}{\subseteq}

\newcommand{\N}{\mathbb{N}}
\newcommand{\Z}{\mathbb{Z}}
\newcommand{\R}{\mathbb{R}}

\begin{center}
  {\LARGE \bf Programmieren I}\\
  {\Large Programmieren in C}\\
  WS 2024
\end{center}

\section{Einführung}
C ist eine imperative Programmiersprache, die bereits seit 1972 existiert und somit eine der ältesten Programmiersprachen ist.
Aufgrund der Eigenschaft, dass C eine Low-Level Sprache ist, wird sie oft für Betriebssysteme, Treiber oder embedded
Systeme wie Arduinos, RasPis oder Autos verwendet.

Anders als bei vielen moderneren Programmiersprachen, gibt es in C relativ wenige Beschränkungen. Prinzipiell macht C genau
das, was man ihm sagt. Das bedeutet aber auch, dass man sich selbst um viele Dinge kümmern muss und es in vielen Fällen
zu ungewollten Nebeneffekten kommen kann, wenn man nicht aufpasst.

\begin{bsp}
  Es existiert ein Array (der Datentyp ist dabei egal) mit 10 Elementen. Um auf ein Element dieses Arrays zuzugreifen,
  wird die Syntax \texttt{array[i]} verwendet, wobei \texttt{i} der Index (also die Stelle) des Elements ist. Das heißt,
  der Aufruf \texttt{array[0]} gibt das erste Element des Arrays zurück, \texttt{array[1]} das zweite, usw.

  Ruft man allerdings das Element \texttt{array[10]} auf, würde man einen Fehler erwarten, da das Array nur 10 Elemente
  beinhaltet. In vielen Programmiersprachen wäre das auch tatsächlich der Fall; Java würde eine \texttt{ArrayIndexOutOfBoundsException}
  werfen, Python einen \texttt{IndexError}, usw. C hingegen verfolgt das Mindset, dass der Entwickler schon weiß, was er
  tut und gibt somit einfach den Wert zurück, der sich an der 11. Stelle des Arrays befinden würde. Das klingt vielleicht
  etwas verwirrend, da es gar keinen 11. Wert gibt, aber ein Array ist im Endeffekt nichts anderes als eine durchgängige
  Reihe an Speicherzellen im Arbeitsspeicher. Über den Index berechnet man die Speicheradresse der jeweiligen Zelle und
  wenn der Index größer als die Länge des Arrays ist, wird auf eine Speicherzelle zugegriffen, die nicht zum Array gehört.
  Daher kann man also auch nicht vorhersehen, welcher Wert in dieser Zelle steht. Das ist zumden natürlich auch ein
  Sicherheitsrisiko, da unbefugt auf fremden Speicher zugegriffen wird und dieser im Wort-Case sensible Daten wie Passwörter
  beinhalten könnte.
\end{bsp}

\section{Hello, World!}
Oft ist das erste Programm, das man in einer (neuen) Programmiersprache schreibt, ein "Hello, World"-Programm. Das ist
ein simples Programm, das den Text "Hello, World!" auf der Konsole ausgibt. Der Sinn dahinter ist, zu überprüfen, ob bei
der Installation der Entwicklungsumgebung, Runtime, usw. alles korrekt verlief oder ob es zu Fehlern kam.

\newpage
Möchte man in C sieht ein solches Programm schreiben, könnte das wie folgt aussehen:
\begin{figure}[h!]
  \ignorespaces
  \begin{minted}{c}
    #include <stdio.h>

    int main() {
      printf("Hello, World!\n");
      return 0;
    }
  \end{minted}
  \caption{Hello, World! in C}
  \label{fig:hello-world}
\end{figure}

Anhand dieses Beispiels lassen sich schon einige Aspekte der generellen C-Syntax erkennen. Zum Beispiel, dass (vereinfacht 
ausgedrückt) fast jede Zeile mit einem Semikolon (\texttt{;}) oder einer geschweiften Klammer (\texttt{\{ \}}) endet. 

\section{Datentypen, Variablen und Operatoren}
Der Code \figref{hello-world} zeigt bereits die grundlegende Struktur eines C-Programms. Der Startpunkt ist die
\texttt{main}-Funktion, welche somit beim Start des Programms aufgerufen wird. Der genaue Aufbau einer Funktion wird in
Kapitel \ref{sec:funktionen} genauer erläutert.

\subsection{Datentypen}
Ein grundlegender Bestandteil von Programmiersprachen sind Datentypen. Mithilfe von Datentypen unterscheidet man, welchen
Wert eine Variable speichern kann. Eine Art von Datentypen sind die \textbf{primitiven Datentypen}. Diese bilden die Basis
für alle anderen Datentypen. Folgende primitive Datentypen gibt es in C:
\begin{table}[h!]
  \centering
  \begin{tabular}{@{}llll@{}}
    \toprule
    Datentyp  & Keyword   & Speichergröße   & Wert             \\ \midrule
    Integer   & int       & 16 bit / 32 bit & Ganze Zahlen     \\
    Short     & short     & 16 bit          & Ganze Zahlen     \\
    Long      & long      & 32 bit          & Ganze Zahlen     \\
    Long Long & long long & 64 bit          & Ganze Zahlen     \\
    Float     & float     & 32 bit          & Gleitkommazahlen \\
    Double    & double    & 64 bit          & Gleitkommazahlen \\
    Character & char      & 8 bit           & Ganze Zahlen     \\ \bottomrule
  \end{tabular}
  \caption{Primitive Datentypen in C}
\end{table}

Für alle ganzzahligen Datentypen kann außerdem ein \texttt{unsigned} vorangestellt werden, um nur positive Werte
zuzulassen und somit den Wertebereich "nach oben" zu verdoppeln.

Die einzelnen Speichergrößen sind zudem abhängig von der verbauten CPU sowie dem verwendeten Compiler. So können
insbesondere die Speichergrößen von Integern und Longs variieren.

\subsection{Variablen}
Wenn man in C von einer Variable spricht, meint man damit eigentlich nur einen fest reservierten Platz im Arbeitsspeicher.
In C haben Variablen außerdem einen festen Datentypen, den man bestimmt, wenn man die Variable \textbf{deklariert} (also dem Compiler
mitteilt, dass diese Variable existiert). Um eine Variable zu deklarieren, schreibt man in C:
\begin{minted}{c}
  <Datentyp> <Variablenname>;
\end{minted}

\begin{bsp}
  Um eine Variable \texttt{a} vom Typ \texttt{int} zu deklarieren, verwendet man in C den Aufruf:
  \begin{minted}{c}
  int a;
  \end{minted}
\end{bsp}

Durch eine reine Deklaration wird der Variable allerdings noch kein Wert zugewiesen. Würde man die Variable nun so aufrufen,
würde sie den Wert zurückgeben, der noch im Speicher an der Stelle vorhanden ist. Daher sollte eine Variable immer
bei der Deklaration einen Wert zugewiesen bekommen:
\begin{minted}{c}
  int a = 0;
\end{minted}

Diese Zuweisung nennt man \textbf{Initialisierung}.

\subsection{Operatoren}
Operatoren bieten unter anderem die Möglichkeit, Variablen miteinander zu verknüpfen. Es gibt dabei verschiedene Arten
von Operatoren. Einige, die man man bereits aus der Mathematik kennt, sind die \textbf{arithmetischen Operatoren}:
\begin{table}[h!]
  \centering
  \begin{tabular}{@{}ll@{}}
    \toprule
    Operator & Bedeutung                    \\ \midrule
    +        & Addition                     \\
    -        & Subtraktion                  \\
    *        & Multiplikation               \\
    /        & Division                     \\
    \%       & Modulo (Rest einer Division) \\ \bottomrule
  \end{tabular}
  \caption{Arithmetische Operatoren in C}
\end{table}

\begin{bsp}
  Gegeben sind zwei Variablen \texttt{a} und \texttt{b} und man möchte den Rest der ganzzahligen Division von \texttt{a}
  durch \texttt{b} berechnen. Der Code dazu könnte wie folgt aussehen:
  \begin{minted}{c}
    int a = 10;
    int b = 3;
    int rest = a % b; // = 1, da 10/3 = 3 Rest 1
  \end{minted}

\end{bsp}

Des Weiteren existieren auch noch \textbf{logische Operatoren}, welche dazu dienen, einen Wahrheitswert aus zwei oder
mehreren Variablen zu schließen. In C gibt es dafür folgende \textbf{logische Operatoren}:
\begin{table}[h!]
  \centering
  \begin{tabular}{@{}ll@{}}
    \toprule
    Operator   & Bedeutung       \\ \midrule
    \&\&       & logisches Und   \\
    $\mid\mid$ & logisches Oder  \\
    !          & logisches Nicht \\ \bottomrule
  \end{tabular}
  \caption{Logische Operatoren in C}
\end{table}

Logische Operatoren werden häufig genutzt, um Bedingungen zu überprüfen. Dazu mehr in Kapitel \ref{sec:kontrollstrukturen}.

Die dritte Art von Operatoren sind \textbf{Vergleichsoperatoren}. Diese verwendet man, um zwei Werte direkt miteinander
zu vergleichen und geben daher ebenfalls einen Wahrheitswert zurück. In C gibt es folgende Vergleichsoperatoren:
\begin{table}[h!]
  \centering
  \begin{tabular}{@{}ll@{}}
    \toprule
    Operator & Bedeutung           \\ \midrule
    $==$     & Gleich              \\
    $!=$     & Ungleich            \\
    $>$      & Größer              \\
    $<$      & Kleiner             \\
    $>=$     & Größer oder gleich  \\
    $<=$     & Kleiner oder gleich \\ \bottomrule
  \end{tabular}
  \caption{Vergleichsoperatoren in C}
\end{table}

\textbf{Wichtig:} \texttt{=} und \texttt{==} sind zwei verschiedene Operatoren! Das einfache Gleichheitszeichen wird verwendet,
um einer Variable einen Wert zuzuweisen. Möchte man den Wert einer Variable mit einem anderen Wert auf Gleichheit prüfen,
verwendet man dazu immer ein doppeltes Gleichheitszeichen.

Ein entsprechendes Codebeispiel zur Anwendung von Vergleichsoperatoren folgt auch hier in Kapitel \ref{sec:kontrollstrukturen}.

\section{Kontrollstrukturen}{\label{sec:kontrollstrukturen}}
Ein Programm, das lediglich aus Variablen und Operatoren besteht, wäre relativ sinnfrei, da es keine Möglichkeit gäbe,
den Verlauf des Programms auf die Eingaben des Benutzers oder auf andere Gegebenheiten anzupassen. Das könnte zum Beispiel
die Berechtigung eines Benutzers sein; ein Programm soll vielleicht andere Funktionen ausführen, wenn es sich bei dem Benutzer
um einen Administrator handelt.

Um solche Bedingungen in den Verlauf des Programms einzubeziehen, gibt es sogenannte \textbf{Kontrollstrukturen}.
Grundsätzlich lassen sich diese in zwei Kategorien unterteilen: \textbf{Bedingte Anweisungen} und \textbf{Schleifen}.

\subsection{Bedingte Anweisungen}
Eine bedingte Anweisung ist ein Codeblock, der nur dann ausgeführt wird, wenn eine bestimmte Bedingung erfüllt wird. Das
Keyword in C dafür lautet \texttt{if} und ist wie folgt definiert:
\begin{figure}[h!]
  \begin{minted}{c}
    if (<Bedingung>) {
      // Code, der ausgeführt wird, wenn die Bedingung erfüllt ist
    }
  \end{minted}
  \caption{if-Statement in C}
  \label{fig:if-statement}
\end{figure}

\begin{bsp}
  Gegeben ist eine Variable \texttt{alter}, die das Alter des Benutzers speichert. Das Programm soll nun einen Text ausgeben,
  wenn der Benutzer volljährig ist. Der Code dazu könnte wie folgt aussehen:
  \begin{minted}{c}
  int main() {
    int alter = 18;
    if (alter >= 18) {
      printf("Du bist volljährig!\n");
    }

    return 0;
  }
  \end{minted}

  Führt man dieses Programm nun aus, erscheint der Text \texttt{Du bist volljährig!} in der Konsole.
  Ändert man den Wert der Variable \texttt{alter} aber beispielsweise auf \texttt{17}, passiert gar nichts.
  Das liegt daran, dass der Code innerhalb des \texttt{if}-Statements wirklich nur dann ausgeführt wird, wenn die
  Bedingung in den runden Klammern erfüllt ist.

  Um das Programm dahingehend zu erweitern, dass auch ein Text erscheint, wenn der Benutzer nicht volljährig ist, gibt es
  in C das Keyword \texttt{else}. Der Code innerhalb eines \text{else}-Blocks wird immer dann ausgeführt, wenn die
  Bedingung im vorangestelltn \texttt{if}-Statement nicht erfüllt ist. Für das obige Beispiel könnte ein passender
  \texttt{else}-Block wie folgt aussehen:

  \begin{minted}{c}
    int main() {
      int alter = 17;
      if (alter >= 18) {
        printf("Du bist volljährig!\n");
      } else {
        printf("Du bist minderjährig!\n");
      }
  
      return 0;
    }
    \end{minted}
\end{bsp}

\section{Funktionen}{\label{sec:funktionen}}
\end{document}