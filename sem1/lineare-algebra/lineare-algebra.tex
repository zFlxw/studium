\documentclass[11pt]{article}
\usepackage[utf8]{inputenc}
\usepackage{amsmath,amsthm,amsfonts,amssymb,amscd}
\usepackage{multirow,booktabs}
\usepackage[table]{xcolor}
\usepackage{fullpage}
\usepackage{lastpage}
\usepackage{enumitem}
\usepackage{fancyhdr}
\usepackage{mathrsfs}
\usepackage{array}
\usepackage{wrapfig}
\usepackage{setspace}
\usepackage{calc}
\usepackage{multicol}
\usepackage{cancel}
\usepackage[retainorgcmds]{IEEEtrantools}
\usepackage[margin=3cm]{geometry}
\usepackage{amsmath}
\usepackage[most]{tcolorbox} \usepackage{xcolor}

\newlength{\tabcont}
\setlength{\parindent}{0.0in} \setlength{\parskip}{0.05in} \usepackage{empheq} \usepackage{framed}

\colorlet{shadecolor}{orange!15}
\parindent 0in
\parskip 12pt \geometry{margin=1in, headsep=0.25in} \theoremstyle{definition}

\newtheorem{defn}{Definition} \newtheorem{reg}{Rule}
\newtheorem{exer}{Exercise} \newtheorem{note}{Note}
\begin{document}
\title{Mathematik I - Lineare Algebra}

\thispagestyle{empty}

\newcommand{\ol}[1]{\begin{enumerate}#1\end{enumerate}}
\newcommand{\ul}[1]{\begin{itemize}#1\end{itemize}}
\newcommand{\li}[1]{\item{#1}}
\newcommand{\equivto}{\Longleftrightarrow}
\newcommand{\N}{\mathbb{N}}

\begin{center}
  {\LARGE \bf Mathematik I}\\
  {\Large Lineare Algebra}\\
  WS 2024
\end{center}

\section{Logik}
\subsection{Begriffe in der Logik}
$
  \begin{array}{l l}
    \text{Aussage:}     & \text{Ein Satz, der in einem gegebenen Kontext eindeutig wahr oder falsch ist} \\
    \text{Konjunktion:} & \text{Eine logische Verknüpfung (z. B. "und", "oder")}                         \\
    \text{Negation:}    & \text{Umkehrung des Wahrheitswertes einer Aussage}                             \\
    \text{Implikation:} & \text{Aus Aussage $A$ folgt Aussage $B$}
  \end{array}
$

\subsection{Logische Gesetze}
$
  \begin{array}{l l}
    \text{De-Morgansche Gesetze:} & \lnot(A\land B) \equivto \lnot A \lor \lnot B            \\
                                  & \lnot(A\lor B) \equivto \lnot A \land \lnot B            \\\\
    \text{Kommutativgesetz:}      & A \land B \equivto B \land A                             \\
                                  & A \lor B \equivto B \lor A                               \\\\
    \text{Assoziativgesetz:}      & A \land (B \land C) \equivto (A \land B) \land C         \\
                                  & A \lor (B \lor C) \equivto (A \lor B) \lor C             \\\\
    \text{Distributivgesetz:}     & A \land (B \lor C) \equivto (A \land B) \lor (A \land C) \\
                                  & A \lor (B \land C) \equivto (A \lor B) \land (A \lor C)  \\\\
    \text{Absorptionsgesetz:}     & A \land (A \lor B) \equivto A                            \\
                                  & A \lor (A \land B) \equivto A                            \\
  \end{array}
$

\subsection{Quantoren}
$
  \begin{array}{l l}
    \text{Existenzquantor:} & \exists n \in \mathbb{N}\qquad\text{Es existiert ein $n$ in der Menge der natürlichen Zahlen, für das gilt \ldots} \\\\
    \text{Allquantor:}      & \forall n \in \mathbb{N}\qquad\text{Für alle Zahlen $n$ in der Menge der natürlichen Zahlen gilt, \ldots}          \\\\
  \end{array}
$

\subsection{Beweisarten}
\subsubsection{Direkter und Indirekter Beweis}
$
  \begin{array}{l l}
    \text{Direkter Beweis:}                                                 & A \longrightarrow B                                                                        \\
    Beispiel\quad n~\text{ist gerade} \longrightarrow n^2~\text{ist gerade} & \exists n \in \N : n = 2k \Longrightarrow n^2 = (2k)^2 = 4k^2 = \underbrace{2(2k)}_{\in\N} \\\\
    \text{Indirekter Beweis (Widerspruchsbeweis):}                          & \lnot A \longrightarrow\text{Widerspruch}                                                  \\
    Beispiel\quad\text{Behauptung:~}\sqrt{2}~\text{ist irrational}          & \text{Annahme: }\sqrt{2}~\text{ist rational}                                               \\
                                                                            & \sqrt{2} = \frac{a}{b} \Longrightarrow 2 = \frac{a^2}{b^2} \Longrightarrow a^2 = 2b^2      \\
                                                                            & \Longrightarrow a^2~\text{ist gerade} \Longrightarrow a~\text{ist gerade}                  \\
                                                                            & \Longrightarrow a = 2k \Longrightarrow 2b^2 = 4k^2 \Longrightarrow b^2 = 2k^2              \\
                                                                            & \Longrightarrow b^2~\text{ist gerade} \Longrightarrow b~\text{ist gerade}                  \\
                                                                            & \Longrightarrow \text{Widerspruch, da }a~\text{und }b~\text{beide gerade sind}             \\
  \end{array}
$

\subsubsection{Beweis durch vollständige Induktion}
Die vollständige Induktion besteht aus folgenden Schritten:
\begin{enumerate}
  \item Induktionsanfang: Zeige, dass die Aussage für ein beliebiges $n$ gilt (meist $n = 0$ oder $n = 1$).
  \item Induktionsvoraussetzung: durch den Induktionsanfang ist bewiesen, dass es mindestens ein $n$ gibt, für das die Aussage stimmt.
  \item Induktionsbehauptung: Es wird angenommen, dass wenn die Aussage für $n$ stimmt, dass sie auch für $n + 1$ stimmen muss.
  \item Induktionsschritt: Beweis, dass die Induktionsbehauptung richtig ist.
\end{enumerate}
Das genaue Vorgehen beim Induktionsbeweis hängt von der konrekten Aussage ab.\\
Beispiel (Gaußsche Summenformel):
\[
  \sum\limits_{k=1}^{n}k = \frac{n(n+1)}{2}
\]
$
  \begin{array}{l l}
    \text{\underline{Induktionsanfang (IA):}}        & n = 1 \quad  1=\frac{2}{2}\quad\checkmark                                          \\\\
    \text{\underline{Induktionsvoraussetzung (IV):}} & \exists n \in \N : \sum\limits_{k=1}^{n}k = \frac{n(n+1)}{2}                       \\\\
    \text{\underline{Induktionsbehauptung (IB):}}    & \sum\limits_{k=1}^{n+1}k = \frac{(n+1)((n+1)+1)}{2} = \frac{(n+1)(n+2)}{2}         \\\\
    \text{\underline{Induktionsschritt (IS):}}       & \sum\limits_{k=1}^{n+1}k = \underbrace{\sum\limits_{k=1}^{n}k}_{\text{IV}} + (n+1) \\\\
                                                     & = \frac{n(n+1)}{2} + (n+1)                                                         \\\\
                                                     & = \frac{n(n+1) + 2(n+1)}{2}                                                        \\\\
                                                     & = \frac{n^2+n+2n+2}{2}                                                             \\\\
                                                     & = \frac{(n+1)(n+2)}{2}\quad \qed
  \end{array}
$
\end{document}