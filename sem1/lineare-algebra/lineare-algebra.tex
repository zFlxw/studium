\documentclass[11pt]{article}
\usepackage[utf8]{inputenc}
\usepackage{amsmath,amsthm,amsfonts,amssymb,amscd}
\usepackage{multirow,booktabs}
\usepackage[table]{xcolor}
\usepackage{fullpage}
\usepackage{lastpage}
\usepackage{enumitem}
\usepackage{fancyhdr}
\usepackage{mathrsfs}
\usepackage{array}
\usepackage{wrapfig}
\usepackage{setspace}
\usepackage{calc}
\usepackage{multicol}
\usepackage{cancel}
\usepackage[retainorgcmds]{IEEEtrantools}
\usepackage[margin=3cm]{geometry}
\usepackage{amsmath}
\usepackage[most]{tcolorbox} \usepackage{xcolor}

\newlength{\tabcont}
\setlength{\parindent}{0.0in} \setlength{\parskip}{0.05in} \usepackage{empheq} \usepackage{framed}

\colorlet{shadecolor}{orange!15}
\parindent 0in
\parskip 12pt \geometry{margin=1in, headsep=0.25in} \theoremstyle{definition}

\newtheorem{defn}{Definition} \newtheorem{reg}{Rule}
\newtheorem{exer}{Exercise} \newtheorem{note}{Note}
\begin{document}

\thispagestyle{empty}

\newtheorem{definition}{Definition}[section]

\newcommand{\ol}[1]{\begin{enumerate}#1\end{enumerate}}
\newcommand{\ul}[1]{\begin{itemize}#1\end{itemize}}
\newcommand{\li}[1]{\item{#1}}
\newcommand{\equivto}{\Longleftrightarrow}

\newcommand{\N}{\mathbb{N}}
\newcommand{\Z}{\mathbb{Z}}

\begin{center}
  {\LARGE \bf Mathematik I}\\
  {\Large Lineare Algebra}\\
  WS 2024
\end{center}

\section{Logik}
\begin{definition}
  Logik ist die \textit{Lehre vom Argumentieren}, bzw. die \textit{Lehre vom Schlussfolgern}.
  Sie hat das Ziel, die Regeln des Argumentierens so streng zu setzen, dass Widersprüche und Paradoxen möglichst
  ausgeschlossen sind.
\end{definition}
\subsection{Begriffe in der Logik}
$
  \begin{array}{l l}
    \text{Aussage:}      & \text{Ein Satz, der in einem gegebenen Kontext eindeutig wahr oder falsch ist} \\
    \text{Konjunktion:}  & \text{Eine logische Verknüpfung (z. B. "und", "oder")}                         \\
    \text{Negation:}     & \text{Umkehrung des Wahrheitswertes einer Aussage}                             \\
    \text{Implikation:}  & \text{Aus Aussage $A$ folgt Aussage $B$}                                       \\\\
    \text{Tautologie:}   & \text{Eine Aussage, die immer wahr ist}                                        \\
    \text{Kontradiktion} & \text{Eine widersprüchliche Aussage, die immer falsch ist}                     \\
  \end{array}
$

\subsection{Logische Gesetze}
$
  \begin{array}{l l}
    \text{De-Morgansche Gesetze:} & \lnot(A\land B) \equivto \lnot A \lor \lnot B            \\
                                  & \lnot(A\lor B) \equivto \lnot A \land \lnot B            \\\\
    \text{Kommutativgesetz:}      & A \land B \equivto B \land A                             \\
                                  & A \lor B \equivto B \lor A                               \\\\
    \text{Assoziativgesetz:}      & A \land (B \land C) \equivto (A \land B) \land C         \\
                                  & A \lor (B \lor C) \equivto (A \lor B) \lor C             \\\\
    \text{Distributivgesetz:}     & A \land (B \lor C) \equivto (A \land B) \lor (A \land C) \\
                                  & A \lor (B \land C) \equivto (A \lor B) \land (A \lor C)  \\\\
    \text{Absorptionsgesetz:}     & A \land (A \lor B) \equivto A                            \\
                                  & A \lor (A \land B) \equivto A                            \\
  \end{array}
$

\subsection{Quantoren}
$
  \begin{array}{l l}
    \text{Existenzquantor:} & \exists n \in \mathbb{N}\qquad\text{Es existiert ein $n$ in der Menge der natürlichen Zahlen, für das gilt \ldots} \\\\
    \text{Allquantor:}      & \forall n \in \mathbb{N}\qquad\text{Für alle Zahlen $n$ in der Menge der natürlichen Zahlen gilt, \ldots}          \\\\
  \end{array}
$

\subsection{Beweisarten}
\subsubsection{Direkter und Indirekter Beweis}
$
  \begin{array}{l l}
    \text{Direkter Beweis:}                                                 & A \longrightarrow B                                                                        \\
    Beispiel\quad n~\text{ist gerade} \longrightarrow n^2~\text{ist gerade} & \exists n \in \N : n = 2k \Longrightarrow n^2 = (2k)^2 = 4k^2 = \underbrace{2(2k)}_{\in\N} \\\\
    \text{Indirekter Beweis (Widerspruchsbeweis):}                          & \lnot A \longrightarrow\text{Widerspruch}                                                  \\
    Beispiel\quad\text{Behauptung:~}\sqrt{2}~\text{ist irrational}          & \text{Annahme: }\sqrt{2}~\text{ist rational}                                               \\
                                                                            & \sqrt{2} = \frac{a}{b} \Longrightarrow 2 = \frac{a^2}{b^2} \Longrightarrow a^2 = 2b^2      \\
                                                                            & \Longrightarrow a^2~\text{ist gerade} \Longrightarrow a~\text{ist gerade}                  \\
                                                                            & \Longrightarrow a = 2k \Longrightarrow 2b^2 = 4k^2 \Longrightarrow b^2 = 2k^2              \\
                                                                            & \Longrightarrow b^2~\text{ist gerade} \Longrightarrow b~\text{ist gerade}                  \\
                                                                            & \Longrightarrow \text{Widerspruch, da }a~\text{und }b~\text{beide gerade sind}             \\
  \end{array}
$

\subsubsection{Beweis durch vollständige Induktion}
Die vollständige Induktion besteht aus folgenden Schritten:
\begin{enumerate}
  \item Induktionsanfang: Zeige, dass die Aussage für ein beliebiges $n$ gilt (meist $n = 0$ oder $n = 1$).
  \item Induktionsvoraussetzung: durch den Induktionsanfang ist bewiesen, dass es mindestens ein $n$ gibt, für das die Aussage stimmt.
  \item Induktionsbehauptung: Es wird angenommen, dass wenn die Aussage für $n$ stimmt, dass sie auch für $n + 1$ stimmen muss.
  \item Induktionsschritt: Beweis, dass die Induktionsbehauptung richtig ist.
\end{enumerate}
Das genaue Vorgehen beim Induktionsbeweis hängt von der konrekten Aussage ab.

\paragraph{Beispiel (Gaußsche Summenformel):}
\[
  \sum\limits_{k=1}^{n}k = \frac{n(n+1)}{2}
\]
$
  \begin{array}{l l}
    \text{\underline{Induktionsanfang (IA):}}        & n = 1 \quad  1=\frac{2}{2}\quad\checkmark                                          \\\\
    \text{\underline{Induktionsvoraussetzung (IV):}} & \exists n \in \N : \sum\limits_{k=1}^{n}k = \frac{n(n+1)}{2}                       \\\\
    \text{\underline{Induktionsbehauptung (IB):}}    & \sum\limits_{k=1}^{n+1}k = \frac{(n+1)((n+1)+1)}{2} = \frac{(n+1)(n+2)}{2}         \\\\
    \text{\underline{Induktionsschritt (IS):}}       & \sum\limits_{k=1}^{n+1}k = \underbrace{\sum\limits_{k=1}^{n}k}_{\text{IV}} + (n+1) \\\\
                                                     & = \frac{n(n+1)}{2} + (n+1)                                                         \\\\
                                                     & = \frac{n(n+1) + 2(n+1)}{2}                                                        \\\\
                                                     & = \frac{n^2+n+2n+2}{2}                                                             \\\\
                                                     & = \frac{(n+1)(n+2)}{2}\quad \qed
  \end{array}
$
\section{Mengenlehre}
\begin{definition}
  Eine Menge ist eine Zusammenfassung von bestimmten, wohlunterschiedenen Objekten unserer Anschauung oder unseres Denkens zu einem Ganzen.
\end{definition}

\subsection{Beispiele von Mengen}
$
  \begin{array}{l l}
    \N = \{1, 2, 3, 4, \ldots\}                             & \text{(Natürliche Zahlen)} \\
    \mathbb{Z} = \{\ldots, -3, -2, -1, 0, 1, 2, 3, \ldots\} & \text{(Ganze Zahlen)}      \\
    \text{M} = \{1, \pi, a\}                                                             \\
    \text{N} = \{x\in\Z | x = k^2\}                                                      \\
    \text{O} = \emptyset\text{ oder }\{\}                   & \text{(Leere Menge)}       \\
  \end{array}
$

\subsection{Definitionen}
Eine Menge $A$ ist eine \textbf{Teilmenge} von $B$, wenn jedes Element von $A$ auch in $B$ enthalten ist.
\[
  A \subseteq B \equivto \forall x(x \in A \Longrightarrow x \in B)
\]

Die \textbf{Schnittmenge} von $A$ und $B$ ist die Menge aller Elemente, die sowohl Teil von $A$ als auch Teil von $B$ sind.
\[
  A \cap B = \{x | x \in A \land x \in B\}
\]
Wenn $A \cap B = \emptyset$, dann heißen $A$ und $B$ \textbf{disjunkt}.

Die \textbf{Vereinigungsmenge} von $A$ und $B$ ist die Menge aller Elemente, die in $A$ oder in $B$ enthalten sind.
\[
  A \cup B = \{x | x \in A \lor x \in B\}
\]

$
  \begin{array}{l l l}
    \text{Sind } A, B, C \subseteq \text{ gilt:}                                                                                \\
    \text{Kommutativgesetz:}  & A \cap B = B \cap A                            & A \cup B = B \cup A                            \\
    \text{Assoziativgesetz:}  & A \cap (B \cap C) = (A \cap B) \cap C          & A \cup (B \cup C) = (A \cup B) \cup C          \\
    \text{Distributivgesetz:} & A \cap (B \cup C) = (A \cap B) \cup (A \cap C) & A \cup (B \cap C) = (A \cup B) \cap (A \cup C) \\
    \text{Absorptionsgesetz:} & A \cap (A \cup B) = A                          & A \cup (A \cap B) = A                          \\
  \end{array}
$

Die \textbf{Differenzmenge} von $A$ und $B$ besteht aus allen Elementen der Menge $A$, die nicht in $B$ enthalten sind.
\[
  A \setminus B = \{x | x \in A \land x \notin B\}
\]
\end{document}